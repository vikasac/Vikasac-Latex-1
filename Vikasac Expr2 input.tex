\documentclass[a5paper,90pt]{article}
\usepackage[top=30mm,bottom=30mm,left=25mm,right=25mm]{geometry}

\title{Vikasac Expr2}
\author{vikasac983 }
\date{June 2024}

\begin{document}

\begin{abstract}
    LATEX (usually pronounced “LAY teck,” sometimes “LAH teck,” and never “LAY tex”) is a format, or collection of macro commands, for TEX, the standard for most professional mathematics and scientific writing.
\end{abstract}
\maketitle

\section{Introduction}

Typesetting journal articles, technical reports, books, and slide presentations.
Control over large documents containing sectioning, cross-references, tables and figures.
Typesetting of complex mathematical formulas.
Advanced typesetting of mathematics with AMS-LaTeX.
Automatic generation of bibliographies and indexes.
Multi-lingual typesetting.
Inclusion of artwork, and process or spot colour.
Using PostScript or Metafont fonts.
\footnote{cit gubbi}
\end{document}

